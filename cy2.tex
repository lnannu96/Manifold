\documentclass[12pt]{article}
\usepackage{amsmath,epsf,amssymb,latexsym,enumerate,cite,shadow,array,color}
\usepackage[ps,dvips,matrix,arrow,frame,import,curve,color]{xy}
%\usepackage[matrix,arrow,frame,import,curve,color]{xy}
\UseCrayolaColors

\setlength{\textwidth}{165mm}
\setlength{\textheight}{215mm}
\setlength{\topmargin}{0pt}
\setlength{\oddsidemargin}{0mm}
\setlength{\evensidemargin}{0pt}

\setlength{\unitlength}{1mm}

\newtheorem{theorem}{Theorem}
\newtheorem{definition}{Definition}
\newtheorem{prop}{Proposition}
\newtheorem{result}{Result}
\newtheorem{conjecture}{Conjecture}
\newtheorem{lemma}{Lemma}

% Generate figures.
\newif\iffigs\figstrue

% Nice script font
%
\DeclareFontFamily{U}{rsf}{}
\DeclareFontShape{U}{rsf}{m}{n}{
  <5> <6> rsfs5 <7> <8> <9> rsfs7 <10-> rsfs10}{}
\DeclareMathAlphabet\Scr{U}{rsf}{m}{n}

% Put preprint number in top-right.
%
\def\pplogo{\vbox{\kern-\headheight\kern -29pt
\halign{##&##\hfil\cr&{%\sc
\ppnumber}\cr\rule{0pt}{2.5ex}&\ppdate\cr}
}}
\makeatletter
\def\ps@firstpage{\ps@empty \def\@oddhead{\hss\pplogo}%
  \let\@evenhead\@oddhead % in case an article starts on a left-hand page
}
%      The only change in \maketitle is \thispagestyle{firstpage}
%      instead of \thispagestyle{plain}
\def\maketitle{\par
 \begingroup
 \def\thefootnote{\fnsymbol{footnote}}
 \def\@makefnmark{\hbox{$^{\@thefnmark}$\hss}}
 \if@twocolumn
 \twocolumn[\@maketitle]
 \else \newpage
 \global\@topnum\z@ \@maketitle \fi\thispagestyle{firstpage}\@thanks
 \endgroup
 \setcounter{footnote}{0}
 \let\maketitle\relax
 \let\@maketitle\relax
 \gdef\@thanks{}\gdef\@author{}\gdef\@title{}\let\thanks\relax}
%\def\thebibliography#1{\section*{References\@mkboth
% {REFERENCES}{REFERENCES}}\small\list
% {[\arabic{enumi}]}{\settowidth\labelwidth{[#1]}\leftmargin\labelwidth
% \advance\leftmargin\labelsep
% \usecounter{enumi}}
% \def\newblock{\hskip .11em plus .33em minus .07em}
% \sloppy\clubpenalty4000\widowpenalty4000
% \hfuzz=1cm
% \sfcode`\.=1000\relax}
%\let\endthebibliography=\endlist
\makeatother

\def\KL{\mbox{KL}}
\def\rem{$\clubsuit$}
\def\E#1#2{\mathbb{E}_{#1}\left[#2\right]}
\def\Enull#1{\mathbb{E}\left[#1\right]}
\def\degree{\mbox{degree}\,}
\def\IC{\mathbb{C}}
\def\IN{\mathbb{N}}
\def\IZ{\mathbb{Z}}
\def\IR{\mathbb{R}}
\def\IP{\mathbb{P}}

\def\CM {{\cal M}}
\def\CN {{\cal N}}
\def\CR {{\cal R}}
\def\CD {{\cal D}}
\def\CF {{\cal F}}
\def\CI {{\cal I}}
\def\CJ {{\cal J}}
\def\CP {{\cal P }}
\def\CQ{{\cal Q}}
\def\CL {{\cal L}}
\def\CV {{\cal V}}
\def\CO {{\cal O}}
\def\CX {{\cal X}}
\def\CZ {{\cal Z}}
\def\CE {{\cal E}}
\def\CG {{\cal G}}
\def\CH {{\cal H}}
\def\CC {{\cal C}}
\def\CB {{\cal B}}
\def\CS {{\cal S}}
\def\CA{{\cal A}}
\def\CK{{\cal K}}
\def\CW{{\cal W}}
\def\CY{{\cal Y}}
\def\CZ{{\cal Z}}
\def\la{\langle}
\def\ra{\rangle}
\def\vev#1{\bigg\langle #1 \bigg\rangle}
\def\half{\frac{1}{2}}
\def\tM{\tilde{M}}
\def\BT{{\bf T}}
%
\newcommand{\secn}[1]{Section~\ref{#1}}
\newcommand{\tbl}[1]{Table~\ref{#1}}
\newcommand{\eq}[1]{Eq.~(\ref{eq:#1})}
\newcommand{\fig}[1]{Fig.~\ref{#1}}
\newcommand{\nl}{\nonumber \\}
%
\newcommand{\be}{\begin{equation}}
\newcommand{\ee}{\end{equation}}
\newcommand{\ba}{\begin{array}}
\newcommand{\ea}{\end{array}}
\newcommand{\bea}{\begin{eqnarray}}
\newcommand{\eea}{\end{eqnarray}}
\newcommand{\sect}[1]{\setcounter{equation}{0}\section{#1}}
%\renewcommand{\theequation}{\thesection.\arabic{equation}}
\renewcommand{\Im}{{\rm Im }}
\renewcommand{\Re}{{\rm Re }}

%\renewcommand{\thefootnote}{\fnsymbol{footnote}}
\def\one{{\hbox{ 1\kern-.8mm l}}}
%\def\ii{{\rm i}}
\def\comm#1#2{\left[ #1, #2\right]}
\def\acomm#1#2{\left\{ #1, #2\right\}}
\def\Var{{\rm Var\,}}
\def\tr{{\rm tr\,}}
\def\sgn{{\rm sgn\,}}
\def\Mb{{\rm~ Mb}}
\def\Gb{{\rm~ Gb}}
\def\Hz{{\rm~ Hz}}
\def\MHz{{\rm~ MHz}}
\def\GHz{{\rm~ GHz}}
\def\Vol{{\rm Vol\,}}
\def\vol{{\rm vol\,}}
\def\p{\partial}
\def\ba{\bar{a}}
\def\bb{\bar{b}}
\def\bc{\bar{c}}
\def\bd{\bar{d}}
%\def\be{\bar{e}}
\def\bz{\bar{z}}
\def\bZ{\bar{Z}}
\def\bW{\bar{W}}
\def\bD{\bar{D}}
\def\bA{\bar{A}}
\def\bB{\bar{B}}
\def\bR{\bar{R}}
\def\bS{\bar{S}}
\def\bT{\bar{T}}
\def\bU{\bar{U}}
\def\bPi{\bar{\Pi}}
\def\bOmega{\bar{\Omega}}
\def\bpartial{\bar{\partial}}
\def\bj{{\bar{j}}}
\def\bi{{\bar{i}}}
\def\bk{{\bar{k}}}
\def\bl{{\bar{l}}}
\def\bm{{\bar{m}}}
\def\btheta{\bar{\theta}}
\def\bpsi{\bar{\psi}}
\def\bF{\bar{F}}
\def\bs{{\bar{s}}}
\def\bt{\bar{t}}
\def\bv{\bar{v}}
\def\bx{\bar{x}}
\def\by{\bar{y}}
\def\bz{\bar{z}}
\def\btau{\bar{\tau}}
\def\bA{\bar{A}}
\def\bB{\bar{B}}
\def\bC{\bar{C}}
\def\bJ{{\bar{J}}}
\def\bK{{\bar{K}}}
\def\bL{{\bar{L}}}
\def\bX{\bar{X}}
\def\bY{\bar{Y}}
\def\bZ{\bar{Z}}
\def\hK{{\hat K}}

\begin{document}
\setcounter{page}0
\def\ppnumber{\vbox{\baselineskip14pt
\hbox{preprint number}}}
\def\ppdate{June 2019} \date{}

 \iffalse
\emailAdd{Tudor.Ciobanu@stonybrook.edu}
\emailAdd{mdouglas@scgp.stonybrook.edu}
\emailAdd{subramanian.lakshminarasimhan@stonybrook.edu}
\emailAdd{yidi.qi@stonybrook.edu}
\fi

\title{\LARGE Numerical CY metrics from holomorphic networks \\[10mm]}
\author{
Michael R. Douglas, 
Subramanian Lakshminarasimhan and Yidi Qi\\[2mm]
{\normalsize Department of Physics and Simons Center for Geometry and Physics, Stony Brook University} 
}

{\hfuzz=10cm\maketitle}

\def\Large{\large}
\def\LARGE{\large\bf}

\vskip 1cm

\begin{abstract}
We try ML inspired ansatzes for computing Ricci-flat K\"ahler metrics.

Dedicated to the memory of Tudor Ciobanu.
\end{abstract}

\section{ Introduction }

K\"ahler manifolds with Ricci flat metrics, known as Calabi-Yau manifolds, are the
first and still the most important starting point for compactification of string theory to produce
realistic physical models.  Their study led to the discovery of mirror symmetry and a great deal
of interesting mathematics.  A brief survey of these topics appears in \cite{CYreview}, and textbooks
on mirror symmetry include \cite{MS1,MS2}.

No closed form expressions are known for these Ricci flat metrics, and it is generally believed that
none exist.\footnote{
But see \cite{Gaiotto:2008cd,Kachru:2020tat} for an analytic approach to the K3 metric.}
One can get an approximate description using methods of numerical general
relativity, as pioneered by Headrick and Wiseman \cite{Headrick:2005ch}.  The special properties 
of K\"ahler geometry lead to many simplifications, starting with the representation of the metric by
a single function.  Donaldson introduced many ideas such as a representation by projective embeddings
and approximation by balanced metrics \cite{Donaldson}.  Subsequent works simplified and improved
the numerical methods \cite{Douglas:2006hz,Douglas:2006rr,Doran:2007zn,Braun:2007sn,Bunch:2008,Seyyedali:2008,Braun:2008jp,Anderson:2009ge,Headrick:2009jz,Anderson:2010ke,Anderson:2011ed}
so that a fairly simple program can get accurate results.

To simplify a bit, the approach used in these works is to expand the K\"ahler potential in a polynomial basis.
As is the case for spectral methods, this is very good for approximating smooth functions with variations on
length scales $1/k$, where $k$ is the order of the polynomials used.  This is sometimes good enough, but by varying
moduli one can easily produce metrics with structure on arbitrarily short scales, say by approaching a singular
limit, as was studied in \cite{Cui:2019uhy}.  And since on a $D$-dimensional manifold, the number of basis functions grows as $\CO(k^D)$, one cannot take $k$ very large (the ``curse of dimensionality'').  Indeed, almost all of these works restrict
attention to manifolds with large discrete symmetry groups to get a manageable number of coefficients.

In recent years, there has been much success with machine learning (ML) inspired approaches to scientific
computation, and especially the solution of PDE's such as Navier-Stokes and the Schr\"odinger equation.
This is a rapidly developing area and we are not aware of a review covering the material we will use,
but a simple illustration of the approach
we follow is \cite{Hoyer2019}.  This work solves problems in structural optimization -- mathematically, the
unknown is a single function in a bounded domain of $\IR^2$ and the solution is the minimum of a nonlocal
energy functional (the integrated stress).  What is novel is to represent the unknown function,
not in terms of finite elements or a Fourier basis, but as the output of a neural network.  This provides
a space of approximating functions parameterized by the network weights, which can be efficiently computed
using technologies developed for machine learning.  Standard numerical optimization applied
to the energy considered as a function of the weights leads to high quality solutions, which are not limited
by the constraints of previous methods.

These methods have only begun to be explored for higher dimensional geometry.
Works which study Calabi-Yau metrics include \cite{Ashmore:2019wzb,Anderson:2020}.
In \cite{Ashmore:2019wzb}, the K\"ahler potential is represented by a gradient boosted decision tree.
The problem is solved at various lower degrees $k$, and the ML model does extrapolation in the degree
(Richardson extrapolation).    The work \cite{Anderson:2020} develops a method more similar to what we will
describe, with separate functions for the metric components and explicitly imposing the K\"ahler condition. 

The idea we propose and study here is to 
represent the K\"ahler potential as the output of a {\bf holomorphic network}, a
feedforward neural network with complex weights, whose inputs are the complex coordinates
or low degree sections, and which uses the activation function $z\rightarrow z^2$, a holomorphic function.
Its outputs are a subspace of the space of holomorphic sections which is parameterized by weight matrices
in a simple and efficiently computable way.  Thus, we can find the metric in this class which is closest to Ricci flat
by numerical optimization of an error function with respect to the weights, much as in ref. \cite{Headrick:2009jz}.

Since each layer doubles the degree of the section, one can in principle work with large values of $k$.
One faces the usual problems of deep networks in ML, such as exploding gradients, but one can use ML
techniques to deal with them, such as a variation on batch normalization.

This formulation of the numerical Calabi-Yau metric problem 
is formally very similar to supervised learning -- both problems amount to fitting a function given
a set of input-output values $(x_i,y_i)$, where $x_i$ is sampled from an input distribution and $y_i$ is known.
Thus we can use a standard ML package (TensorFlow) to construct our networks and do the optimization.
Our code is described in \S \ref{ss:code} and is available at {\tt https://github.com/yidiq7/Manifold} .

We obtain results for a variety of quintic hypersurfaces, including the familiar
one-parameter family of Fermat quintics, and also hypersurfaces with much less symmetry.
Our main focus is to understand the accuracy of these metrics and its dependence on parameters,
both degree $k$ and the moduli and amount of symmetry of the manifold.

A brief summary of the approach and the mathematical questions it raises can be found in \cite{Douglas2020}.

Polynomial activation functions have appeared in the ML literature \cite{Chrysos2020,Mannelli2020},
but our motivations (holomorphy and contact with higher dimensional geometry) are new.  

\section{ Numerical methods inspired by machine learning }

In this section we alternate between the points specific to our problem and review of the concepts
we use from machine learning.  Some useful background material includes \cite{GH,Huybrechts} on
complex geometry and \cite{Bishop} on machine learning.  We also review the basic definitions we use from
complex geometry in Appendix A.

\subsection{ Review of the embedding method }

Following Donaldson \cite{Donaldson}, most work on numerical Calabi-Yau metrics
represents the metric using an embedding by holomorphic sections of a very ample line bundle $\CL$.
Let us briefly review this.

We recall that a line bundle is defined by choosing patches $U_\alpha$ on the manifold $M$ and holomorphic transition
functions $f_{\alpha\beta}$ on the overlaps of patches $U_\alpha\cup U_\beta$ satisfying the
consistency conditions $f_{\alpha\beta}f_{\beta\gamma}f_{\gamma\alpha}=1$.  A section $s$ is then a holomorphic
function $s_\alpha$ on each patch satisfying $s_\alpha=f_{\alpha\beta}s_\beta$.  Now this is ambiguous
considered as a function, because we can always multiply by a set of holomorphic functions $\lambda_\alpha$
defined on each patch, taking $s_\alpha\rightarrow \lambda_\alpha s_\alpha$.  But the ratio of a pair of sections is
unambiguous.  This can be rephrased as the statement that a vector of $N$ sections is an unambiguous map $\iota$
from $M$ to $\IC\IP^{N-1}$, since $\lambda_\alpha$ acts to rescale the entire vector.
The very ample condition then implies that the map $\iota$ is an embedding.

Pulling back the Fubini-Study metric on $\IC\IP^{N-1}$, 
the embedding then leads to the K\"ahler potential
\be \label{eq:a1}
K = \log \sum_{I,\bJ} h_{I,\bJ} s^I \bs^\bj
\ee
where $s^I$ is a basis of $N=h_0(\CL)$ holomorphic sections.  This gives us an $N^2$ real dimensional
family of metrics parameterized by the hermitian matrix $h_{I,\bJ}$.
One then does integrals over $M$ by Monte Carlo, as in \cite{Douglas:2006rr}.
This setup can be used in various ways, but conceptually the simplest is to choose a measure of how
close the metric is to Ricci flatness and numerically optimize it, as done in \cite{Headrick:2009jz}.
This is easy to program (as we will review
below), especially compared with methods that use explicit coordinate patches or a grid.

Let us focus on the case of $M$ a hypersurface $f=0$ in a weighted projective space $W$.
The homogeneous coordinates will be $Z^i$ with degrees $n_i$.
Now a holomorphic section is simply
a weighted homogeneous polynomial, so this ansatz is easy to work with mathematically.
Physically one can define a holomorphic section as a 
ground state of a magnetic Schr\"odinger operator, {\it i.e.} a state in the lowest Landau level
(see for example \cite{Douglas:2008pz}).
Thus, in the terminology of numerical methods, this is a spectral method.  A spectral method expands
continuous functions in terms of the low lying eigenfunctions of some operator, such as the
translation operator (leading to a Fourier basis) or a Laplacian.  The upper cutoff on eigenvalue
corresponds to a fixed lower cutoff on the length scale, and thus 
spectral methods are usually not well adapted to problems with multiple scales.

In our problem, 
let us denote the degree of the polynomial (equivalently of the line bundle) by $k$.  An expansion
in polynomials of degree $k$ can represent functions with up to $k$ nodes, so with variations on
length scales roughly down to $1/k$.  However,
by adjusting moduli it is easy to produce CY metrics with multiple scales -- for example,
we might have a small resolved cycle in a large bulk manifold.  Thus we would like a more flexible
ansatz than \eq{a1} but one which is still easy to program.  

One option would be to use a higher
degree line bundle and a subset of its sections, keeping more sections in regions with more variation.
Mathematically, this amounts to restricting the rank of the matrix $h_{I\bJ}$.
This could be done by imposing the form $h=U^\dag \lambda U$ with $\lambda$ a
diagonal matrix of restricted rank.
Let us denote the space of metrics of the form \eq{a1} 
with a degree $k$ embedding and in which $\mbox{rank}\, h \le D$ as $\CG_{k,D}$.
It has real dimension $\CO(ND)$ so this mitigates the curse of dimensionality, however to use this truncation we
have to search for an appropriate subspace of sections, which is not straightforward.  Thus we leave this
interesting idea for future investigation.

\subsection{ Feed-forward networks }

Let us briefly review the definition of a feed-forward network (FFN, also called
MLP for multilayer perceptron).  It is
a parameterized space of functions
\be \label{eq:defF}
F_w : \CX \rightarrow \CY ,
\ee
with an input $x\in\CX \cong \IR^D$ and an output $y\in \CY\cong \IR^{D'}$ (we will generally take $D'=1$).
It is the composition of a series of functions or layers indexed $0,1,\ldots,d$.  The number
of layers $d+1$ is the depth.

The parameters or weights $w$ are a series of matrices $\{ w_{(0)},w_{(1)},\ldots,w_{(d)} \}$ which enter the 
following expression:
\bea \label{eq:defFy}
y_j(w;x) &=& \sum_{i_d=1}^{D_d} w^{(d)}_{j,i_d} \theta( z_{(d-1)}^{i_d} )\\
 &\vdots & \nonumber \\ \label{eq:defF2}
  z_{(1)}^{i_2} &=& \sum_{i_1=1}^{D_1} (w_{(1)})^{i_2}_{i_1} \theta( z_{(0)}^{i_1} )\\
  \label{eq:defF1}
 z_{(0)}^{i_1} &=& \sum_{i=1}^D (w_{(0)})^{i_1}_i x^i \\
\eea
The ``activation function'' $\theta(x)$ is a nonlinear function, for which a popular choice is
the ``ReLU'' function 
\be
\theta_{ReLU}(x) = \begin{cases} x, x\ge 0 \\ 0, x<0 \end{cases}.
\ee
The term ``unit'' is sometimes used to denote a sum followed by a single application of $\theta$,
so this network will have $D_1+\ldots+D_d$ units.

It has been shown that feed-forward networks can approximate arbitrary functions.
This is the case even for depth two ($d=1$) \cite{Cybenko}, but in this case one can need an exponentially
large number of units, as would be the case for simpler methods of interpolation (the ``curse of dimension'').
By using more layers, one can gain many advantages -- complicated functions can be represented with
many fewer units, and local optimization techniques are much more effective.  
How exactly this works is not
well understood theoretically and there are many interesting observations and hypotheses as to how these
advantages arise.

As an aside, one
basic application of this in physics and statistics is to use a feed-forward network with a single
output to represent a family $P_w$ of probability distributions on $\CX$ parameterized by the weights $w$,
by taking the output $y$ to be the probability density function.\footnote{
Another very common choice for representing a probability distribution over a finite set
$S$ is to use a network with a separate output $y_a$ for each option $a\in S$, and take the final layer
to be the ``softmax'' $P(a)=\exp y_a/\sum_{b\in S} \exp y_b$ to get a normalized probability distribution $P(a)$.}
The expectation value of a function $\CO(x)$ in the distribution $P_w$ is then
\be
\E{w}{\CO} \equiv \frac{ \int d\mu(x)\ y(w;x)\, \CO(x) }{ \int d\mu(x)\ y(w;x) } .
\ee
Here $d\mu(x)$ is a measure on $\CX$, which mathematically is required to turn the function $y(w;x)$ on $\CX$
into a density on $\CX$.  Given coordinates $x^i$ on $\CX$, this could be Lebesgue measure, but we are free
to make other choices.

In a high dimensional configuration space $\CX$, one often does these integrals by sampling,
choosing a set of $N_p$ points $x_{(i)}\in\CX$ and taking
\be
d\mu(x) = \frac{1}{N_p} \sum_{i=1}^{N_p} \delta(x-x_{(i)}) .
\ee
The sample points $x_{(i)}$ could be distributed 
according to an {\it a priori} probability measure $Dx$ on $\CX$, or we could
adapt the sampling to prefer regions with large measure (importance-based sampling).  Many options have been
developed in statistics.

An analogous idea could be used to get a parameterized family of quantum
wave functions, in this case taking all of $(x,w,y)$ to be complex, and taking
\be
\E{w}{\CO} \equiv \frac{ \int d\mu(x)\ |y(w;x)|^2 \CO(x) }{ \int d\mu(x)\ |y(w;x)|^2 } .
\ee
One would also have to choose an
appropriate $\theta(x)$.   % Some works using these wave functions include \cite{}.

\subsection{ Multilayer holomorphic embeddings }

The idea we will pursue in this work is to use \eq{defFy} to define a subspace of sections.
Thus we take the input space $\CX$ to be the ambient weighted projective space, and we choose
$\theta(x)$ to be a nonlinear homogeneous holomorphic function.  
The simplest choice and the one we will use is to take
\be \label{eq:deftheta}
\theta(x) = x^2 .
\ee
We then take the successive layers \eq{defF1}, \eq{defF2}, {\it etc.} to define subspaces of a space of
sections. 
To get a real valued K\"ahler potential, we replace \eq{defFy} with
\bea \label{eq:defFK}
K(w;Z) &=&  \log\sum_{i_d=1,\bj_d=1}^{D_d} h^{(d)}_{i_d,\bj_d} z_d^{i_d} \bz_{d}^{\bj_d} \\
z_d^{i_d} &\equiv& \theta( z_{(d-1)}^{i_d} ) 
\eea
where $z_{(d-1)}$ and the previous layers are as above.

This construction gives us a class of metrics for each choice of depth $d$ and layer widths $D_1,\ldots,D_d$,
obtained from embeddings with degree  $k=2^d$.  The total number of real weights is
\be
N_w = 2\left( D D_1 + D_1 D_2 + \ldots + D_{d-1} D_d \right) + D_d^2 .
\ee
Generally $D_i < h_0(2^i)$ so this will not span the complete basis of sections, in other words
we have restricted the embedding and are only using a subset of metrics.  While the final layer $z_d$ is a linear
subspace of $H^0(\CL^{2^d})$, this subspace is nonlinearly parameterized by the weights.
The hope is that it is flexible enough to describe the metrics of interest.

A variation on this is to take the inputs to be a complete basis of sections $s^I$ of degree $k_0$, so
\be  \label{eq:defF2}
 z_{(0)}^{i_1} = \sum_{I=1}^{h_0(k_0)} (w_{(0)})^{i_1}_i s^I 
\ee
where $I$ is an index for the basis.  For the hypersurface $f=0$ in $\IC\IP^{D-1}$, the basis will be the symmetrized 
degree $k_0$ monomials, quotiented by the ideal generated by $f$ (if $k_0 \ge \deg f$).

Other combinations of layers and activation functions are of course possible, subject to the constraint
that every activation (intermediate value) is homogeneous (a section of a definite line bundle).
Thus one cannot have skip connections, but one could take other products of outputs from previous layers.

\subsection{ Biholomorphic embeddings }

Another variation is to use biholomorphic sections, meaning products of holomorphic and antiholomorphic sections,
as the activations (intermediate values) of the network.  Thus, we would take
\bea  \label{eq:defF3}
 x_{(0)}^{i_1} &=& \sum_{i,j=1}^D (w_{(0)})^{i_1}_{i,\bj} z^i \bz^{\bj} \\
 x_{(1)}^{i_2} &=& \sum_{i=1}^{D_1} (w_{(1)})^{i_2}_{i_1} \theta(x_{(0)}^{i_1}) \\
& \vdots & \\
 K(w;Z) &=&  \log \sum_{i=1}^{D_d} (w_{(d)})_{i} \theta( x_{(d-1)}^i ) .
\eea
The first layer weights $w_{(0)}$ should be a hermitian matrix, so that the activations and subsequent
weight matrices will all be real.  One still needs the activation function $\theta$ to be homogeneous,
and $z\rightarrow z^2$ is still the natural choice.  The total number of real weights is now
\be
N_w = D^2 + D D_1 + D_1 D_2 + \ldots + D_{d-2} D_{d-1} + D_d.
\ee
Again, there are many variations on this.

A potential advantage of this representation over the holomorphic network is that the higher degree
space of K\"ahler potentials contains the lower degree spaces in a simpler way.  For example, suppose we wanted to
represent
\be
K = \left( \sum_{i=1}^D |z_i|^2 \right)^n
\ee
with $n=2^d$.  This is a degree $n$ K\"ahler potential which leads to exactly the same metric as a degree $1$
potential (up to an overall rescaling).  Representing it with the holomorphic network requires exponentially
wide intermediate layers, but the biholomorphic network can do it with width 1 intermediate layers.


\subsection{ Generalizations to other ambient spaces; discrete symmetries }

A sensible network must respect the gradings etc. 

For hypersurfaces in weighted projective space, we need to modify the construction.
The simplest option is to take the inputs to be a basis of all sections of the same degree as the
defining function $f$.  Another option is not to have a single feed-forward structure but rather
a DAG of layers connected in all ways consistent with homogeneity.  One might also generalize
the activation function, allowing two layers with outputs $z$ and $z'$ to send the product $zz'$
to the next layer.

Next let us consider toric varieties.  In physics terms, a toric variety is the moduli space of a
$U(1)^r$ gauge theory with four supercharges (say, a $(2,2)$ sigma model).  There are many
equivalent mathematical definitions; the closest to the physics is that it is the symplectic (or GIT) quotient
of $\IC^N$ by a $U(1)^r$ action.  Thus a toric manifold $M$ is specified by 
an $N\times r$ integral matrix $\Sigma$ of $U(1)^r$  charges and a moment map, a real-valued function on the
Lie algebra of $U(1)^r$.  If it is a smooth manifold, it will have $b^2=r$, and the fields (the coordinates
on $\IC^N$) are sections of line bundles with $c_1(\CL)$ given by the corresponding row of $\Sigma$
(say this in a more mathematical way).  We can then multiply coordinates and take weighted sums in any way that
preserves the grading, to get a nonlinearly defined subspace of the space of sections of some $\CL$.
All of these constructions should be unified in some representation theoretic way.

A basic example is to take all of the monomials in the defining polynomial, {\it i.e.} a basis of sections of $\CN$,
project to a subspace, and take the sum of the squares of the resulting functions.  Thus we are comparing the
representing power of $\mbox{Sym}^2 H^0(\CL)$ with $\sigma(P\cdot H^0(\CL)$ for some projector $P$.

How can we compare the general ability to represent functions of a basis and a subset?
We can define the associated kernels, which here are the Bergman kernels.  Given a metric on
$\CL$ and a volume form, we can define the inner product on sections and the corresponding Bergman
kernel
\be
\rho(z,\bar z') = \sum_i s(z)\bar s(\bar z') .
\ee
We can then look at $\rho_1-\rho_2$ and show that it has small norm in some sense.
This will probably not be the case, but $\rho_2$ is parameterized and what we really care about is
how well the best $\rho_2$ can approximate $\rho_1$.


\subsection{ Supervised learning, sampling and data }

So far our discussion has not had much to do with machine learning.  Let us briefly review a simple
ML problem, supervised learning, to see the similarities and differences.  It turns out that our problem is similar enough
to this that we could try to adapt standard ML software to do our computations, however as we will
explain there are some technical problems in doing this.

In supervised learning, we have a data set of $N_{data}$ items, each of which is an input-output pair $(x_n,y_n)$.
These are supposed to be drawn from a probability distribution $\CP$ on $\CX\times\CY$.
The goal is to choose the function \eq{defF} from $\CX$ to $\CY$
which best describes the general relation $\CP$ between input and output,
in the sense that it minimizes some definition of the expected error (an objective or ``loss'' function).
The procedure of making this choice given the data set (and perhaps limited ``prior'' information about $\CP$)
is called training the network.

A simple choice of objective function is the mean squared error, 
\be \label{eq:mse}
\CE = \E{\CP}{ \left(f_w(x) - y\right)^2 }.
\ee
If we estimate this by evaluating it on our data set, we get the training error
\be \label{eq:train}
\CE_{train} =  \frac{1}{N_{data}} \sum_{n=1}^{N_{data}}{ \left(f_w(x_n) - y_n\right)^2 }.
\ee

A standard ML training procedure is the following.  We start with an FFN as in \eq{defFy},
with the weights initialized to random values -- in other words, we draw the $w$ from some distribution
independent of the data.  A common choice is for each matrix element $w_{i_m}^{(m),i_{m+1}}$ to be
an independent Gaussian random variable with mean zero and variance $1/\sqrt{D_m}$.  This choice is made so that
the expected eigenvalues of the weight matrix are order $D_m^0$.

The next step is to minimize \eq{train} as a function of the weights.  A simple algorithm for this is
gradient descent, a stepwise process in which the weights at time $t+1$ are derived from those at $t$ as
\be \label{eq:gd}
w({t+1}) = w({t}) - \epsilon(t) \frac{ \partial \CE_{train} } { \partial w }\bigg|_{w=w(t)} .
\ee 
While this will only find a local minimum, it works better for these problems than one might have thought.
One trick for improving the quality of the result is
to make the step size $\epsilon(t)$ decrease with time, according to a ``learning schedule''
chosen empirically to get good results for the task at hand.

An improvement on this procedure is ``stochastic gradient descent'' or SGD.  This is much like \eq{gd}
except that instead of evaluating the training error $\CE_{train}$ on the full data set, one evaluates it on
a subset or ``batch,'' with the batch varied from one step to the next so that their union covers the full data set.  This was
originally done for computational reasons but it also turns out to produce a noise term with beneficial 
properties, for example in helping to escape local minima.  

Finally, once the optimization is deemed to have converged, one judges the results by estimating \eq{mse}.
This estimate must be made by using an independent data set from that used in training as otherwise we are
rewarding our model for matching both signal and noise.\footnote{In classification problems, one often uses networks
with many more parameters than data points and which can completely fit the dataset, so that the minimum
of $\CE_{train}$ is zero!  In this case $\CE_{train}$ is clearly a poor estimate for $\CE$.}
However in most applications we do not have any direct access to $\CP$, rather we only have an empirical
data set.  Thus one starts by dividing the full data set into disjoint ``training'' and ``testing'' subsets, evaluates
\eq{train} on the training set for training, and then evaluates the sum of errors over the testing set to estimate $\CE$. 
The final model can be very accurate, surprisingly so when compared to expectations from standard
statistical theory.  Let us cite \cite{} as a few papers which study these theoretical questions.

While our problem is not one of supervised learning, it will be useful to phrase it in terms as similar
as possible, so that we can most easily use ML software.  The workflow of the supervised learning task involves defining
a set of data points $(x_n,y_n)$ which are independent of the weights, repeated evaluation of the network at each $x_n$
to get a prediction $f(x_n)$ for the corresponding $y_n$, and optimization of an objective function which is
a sum of terms which each depend on a single data point.
The network is normally defined by concatenating layers, such as multiplication by a weight matrix
(a fully connected layer), application of an activation function, and so on.  These layers are implemented
in associated software libraries, such as Keras for Tensorflow.  As we explain next,
while we will have to implement some new layers for our problem, otherwise our workflow is the same.

\subsection{ Batch normalization }

A very common problem with deep networks is the exploding/vanishing gradient problem.
To see this, consider the gradient of the objective function \eq{mse},
\be
\frac{ \partial  } { \partial w }\CE = 2\E{\CP}{ \left(f_w(x) - y\right)\cdot \frac{ \partial f_w(x) } { \partial w } }.
\ee
Using the chain rule, the gradients of $f$ for a given layer
can be written as a product of the gradient of the subsequent layers with respect to their inputs,
times the output of the preceding layers.
These products include a weight at each layer, so one has the general scaling $\nabla \CE \sim w^{d-1}$.
Using this in gradient descent, one has $\dot w \sim w^{d-1}$, which tends to drive the weights to
infinity or zero.

One of the most successful solutions to this problem is batch normalization \cite{Ioffe2015}.
Rather than operate directly on the weights, the basic idea is to ``whiten'' the inputs to each layer.
For each component of the input, one subtracts its mean value
mean and rescales by the variance to standardize its distribution.  This removes the dependence on
the scales of weights from other layers.  Furthermore, it
is a differentiable transformation
and thus one can optimize a network with such a layer.   

Batch normalization has two further ingredients.
First, additional learned weights ($\beta,\gamma)$
are provided to compensate for the standardization.  Explicitly, if the
standardized data points are $\hat x_i$, one uses instead $\gamma\hat x_i+\beta$, with the new weights
optimized along with the original ones.
Second, the standardization is done on a per-batch basis, again to reduce the computational requirements.

Does batch normalization, either in its standard definition or some variant, make sense in our setup?
For us, an overall rescaling of the vector of sections by a local holomorphic function $\lambda$ is allowed,
but not individual rescalings of the components.  Furthermore, shifts are not allowed.
Thus standard batch normalization is not sensible, but we can take the same rescaling for every component,
\be
z_{(n)}^i \rightarrow \lambda(z_{(n)}) z_{(n)}^i .
\ee 
Here $\lambda$ could be a general function, but the natural choice in order to normalize the vector is
\bea
z_{(n)}^i \rightarrow \hat z_{(n)}^i &=& \frac{\gamma}{\sqrt{\sigma^2+ \epsilon}} \cdot z_{(n)}^i \\
\sigma^2 &\equiv & \E{\mbox{batch}}{\sum_i |z_{(n)}^i|^2 }.
\eea
This could be done per-batch, or even on a per point basis.  
The weight $\gamma$ is redundant with the next layer weights, but
using it changes the optimization problem, so we might include it.
The parameter $\epsilon$ is included for numerical stability and should be small.

Finally, the backpropagation through this layer uses
\bea
\frac{ \partial \hat z_{(n)}^i }{ \partial z_{(n)}^j } &=& \frac{\gamma}{\sqrt{\sigma^2+ \epsilon}} \delta^i_j -
\frac{\gamma}{2(\sigma^2+ \epsilon)^{3/2}} z_{(n)}^i  \frac{ \partial \sigma^2 }{ \partial z_{(n)}^j } \\
&=& \frac{\gamma}{\sqrt{\sigma^2+ \epsilon}} - \frac{\gamma}{2(\sigma^2+ \epsilon)^{3/2}} z_{(n)}^i \bz_{(n)}^j .
\eea
There is a similar expression for the biholomorphic network in terms of the real quantities $x_{(n)}^i$,
\be
\frac{ \partial \hat x_{(n)}^i }{ \partial x_{(n)}^j } =
\frac{\gamma}{\sqrt{\sigma^2+ \epsilon}} - \frac{\gamma}{(\sigma^2+ \epsilon)^{3/2}} x_{(n)}^i x_{(n)}^j .
\ee
Presumably, we can modify the Tensorflow batch normalization layer to do this.

\section{ Implementation }


\subsection{ Geometric quantities } 

There is the holomorphic three-form
\be
\Omega = \ldots 
\ee
and the associated volume form
\be
d\mu_\Omega \equiv \CN_\Omega \Omega \wedge \bar\Omega .
\ee
The normalization $\CN_\Omega$ will be described shortly.  
This volume form depends on the complex structure but is independent of the K\"ahler form
and thus the embedding we use to represent $M$.

The Kahler form and associated volume element are
\bea \label{eq:omegag}
\omega_g &=& \partial_i \partial_\bj K \\
\label{eq:MA}
d\mu_g \equiv \omega_g^3 &=& \det \omega_g = \det_{i,\bj} \partial_i \partial_\bj K .
\eea
Now the integral $\int_M\omega_g^3$ is a topological invariant, so one can choose the normalization
of $d\mu_\Omega$ to make $\eta=1$ for the Ricci flat metric.  

\subsection{ General implementation }

This follows the approach taken in \cite{}
with the main difference being the use of the feed-forward network \eq{defFK}.

We start by sampling points on $M$ for doing integrals, using the procedure of \cite{Douglas}.
This sample will play the role of the input dataset in a supervised learning task.
It will be naturally distributed according to the Fubini-Study volume element and thus (using a theorem
of Shiffman-Zelditch)
\be
\lim_{N_{data}\rightarrow\infty} \frac{1}{N_{data}} \sum_i f(x_i) = \int_M \omega_{FS}^3 f(x) ,
\ee
where the Fubini-Study metric $\omega_{FS}$ and volume element is the special case of \eq{omegag} with $k=1$
and a fixed hermitian metric $h_{i\bj}$.  Given such a metric, to sample from $M$ we first choose two points
$a$ and $b$ on the ambient WP space\footnote{ 
More details on computations on WP can be found
in the Penn group papers.} 
from a normal distribution with covariance $h_{i\bj}$.
We then find the points on the intersection of the line $\lambda a+\rho b$ with $M$, in other words
the choices of $\lambda/\rho$ for which $f(\lambda a+\rho b)=0$.  This equation will have $\deg f$
solutions (with multiplicity) and for our purposes (which do not look at correlations between points) we
can simply add all of them to the data set as inputs $x_n$, evaluating the variable $y_n$ for each.

By reweighting, we define an empirical approximation to $d\mu_\Omega$, which is
\be\label{eq:muOmegaD}
d\mu_{\Omega,\CD} = \sum_{i\in \CD} \frac{ |\Omega(x_i)|^2 }{ \omega_{FS}(x_i)^3 } .
\ee

The objective function measures the deviation of the metric from Ricci flatness.  The simplest such measure
is
\be
\eta = \frac{ \omega_g^3 }{ \Omega\wedge\bar\Omega }
\ee
The ``least squares error'' for Ricci flatness is then
\bea \label{eq:Lquad}
\CE &=& \int_M \Omega\wedge\bar\Omega \left(\eta-1\right)^2 \\
&=& \int_M \omega_{FS}^3 \frac{\left( \omega_g^3-\Omega\wedge\bar\Omega \right)^2 }{ 
\omega_{FS}^3 \Omega\wedge\bar\Omega } \\
&=& \int_M \omega_{FS}^3 \frac{(f_w(x_i) - y_i)^2}{y_i}
\label{eq:cyobj}
\eea
where
\bea
y_i &=& \frac{ \Omega\wedge\bar\Omega }{ \omega_{FS}^3 } \\
f_w(x_i) &=& \frac{ \omega_g^3 }{ \omega_{FS}^3 } .
\eea
The point of writing it this way is that $y_i$ is independent of the weights, so we can also consider
it as part of the ``data set,'' the target value for $f_w$ evaluated at the point $x_i$.  An ML package
will provide common objective functions such as least squared error and we will be able to use this one directly.

Note that we can choose a different function of $\eta$ which has the same optimum yet has
a statistics/information theory interpretation.  For example, if we consider a volume form on the CY as
a probability density, it is natural to look at the KL divergence:
\be \label{eq:KLobj}
\mbox{KL}\left( d\mu_\Omega | d\mu_g \right) = 
-\int_M d\mu_\Omega \log \eta .
\ee
This will also have its minimum at the Ricci-flat metric.
Still, it may be less convenient for optimization, so we compute with the least squares definition.

Thus we need to construct layers which implement \eq{defFK}, the two derivatives which produce $g_{i\bj}$,
and the nonlinear definition of $f_w$ in terms of this.  Now optimization requires evaluating the gradient
of $f_w$ and this is done by the backpropagation algorithm built into the ML package.  Thus there is already
a built-in ability to compute derivatives, the question is how to use it to compute the three derivatives
implicit in the definition $\mbox{grad}\, \CE$.  This should be possible because we can chain gradient
computations, for example see \cite{TFhessian}.

Now given that the inputs to the network are the coordinates $Z^a$ of a point $x_i$, the derivatives with
respect to $Z$ can be obtained by interpolating an additive layer $Z\rightarrow Z+b$ and taking the $b$
derivative.  Presumably, we can combine these to get the required second and third derivatives.


\subsection{ Adaptive method }

We could vary the dataset to lower the variance of the objective function.  Interesting to work out
but not clear it is worth implementing.

\section{ Results }

\subsection{ Examples }

\begin{enumerate}
\item The Fermat quintics \eq{fermat}.
\item A two parameter family with less symmetry,
\be
f_1 = z_0^5 + z_1^5 + z_2^5 + z_3^5 + z_4^5 + \psi z_0z_1z_2z_3z_4 +\phi (z_3z_4^4 + z_3^2z_4^3 + z_3^3z_4^2 + z_3^4z_4)
\ee
\item Another two parameter family,
\bea
f_2 &=& z_3 g(z) + z_4 h(z) \\
g &=& z_0^4 + z_1^4 + z_2^4 + z_3^4 + a z_0z_1z_2z_3 \\
h &=& z_0^4 + z_1^4 + z_2^4 + z_4^4 + b z_0z_1z_2z_4 \\
\eea
\item Possibly, some randomly chosen $f$'s.
\end{enumerate}

Each should be worked out with
\begin{enumerate}
\item  The standard space of metrics \eq{a1} for as many $k$'s as we can reasonably do.
\item A two layer holomorphic network, where the inputs are the coordinates.
\item A two layer holomorphic network, where the inputs are quadratic ($k=2$) and cubic ($k=3$) in the coordinates.
\item Two layer biholomorphic networks with inputs $z\bz$ and $z^2\bz^2$.
\item Maybe, a few examples of three layer networks, perhaps with batch norm.
\end{enumerate}

\subsection{ Observations }

For each of these, we want the training and testing $\eta$ and $\delta\eta$, as we are computing now.
We probably also want the maximum $|\eta-1|$.

Graph these as functions of the number of layers and number of parameters.
One fair way to compare all of the different networks is to make the x-axis the total number of real parameters.

\subsection{ Interpretation }

There are two geometric properties of the CY which we expect to control the accuracy.
One is the smallest length scale on the CY.  The other is the ``complexity'' of the manifold,
a somewhat vague concept, but somehow inversely related to the amount of symmetry.
Both of these properties are hard to measure and thus we look at measurable proxies.

A reasonable proxy for the smallest length scale is the distance in moduli space to the nearest
singular Calabi-Yau manifold, for which $f=\nabla f=0$ has a simultaneous solution in $\IC\IP^4$.
For example, for the Fermat family of quintics \eq{fermat}, these equations are \eq{fermat} and
$ 5 z_i^5 = - \psi z_1z_2z_3z_4z_5 \, \forall 1\le i\le 5$, which has a solution if $\psi=-5\omega$
for $\omega$ any fifth root of unity.  The limit $\psi\rightarrow\infty$ is also singular.

The generic singularity of a hypersurface is an ordinary double point (ODP).  In a small neighborhood
of a $D=3$ ODP singularity it looks like
\be\label{eq:conifold}
z_1^2 + z_2^2 + z_3^2 + z_4^2 = \epsilon
\ee
and the singular limit is $\epsilon\rightarrow 0$.  This is usually called the conifold singularity in the
string theory literature.  A Ricci-flat metric on this noncompact manifold is known \cite{Candelas1990}
and it looks like the total space of $T^*S^3$, with the volume of the $S^3$ shrinking to zero as $\epsilon^{3/2}$
when $\epsilon\rightarrow 0$.\footnote{This can be seen by writing the real and imaginary parts of
\eq{conifold} separately.  For $\epsilon>0$ real, the $S^3$ is the submanifold $\Im z_i=0$.}
Thus the smallest length scale on this noncompact CY is $L \sim \epsilon^{1/2}$.
For $\epsilon\ne 0$, the solution of $\nabla f=0$ is all $z_i=0$, and the distance (in the Euclidean metric)
from this point to the closest solution of $f=0$ is also $\epsilon^{1/2}$.  

The coefficients of the defining function $f=0$ parameterize the complex structure moduli space $\CM_c$ of quintic
hypersurfaces, up to an overall $GL(5)$ action.  The submanifold of $\CM_c$ on which the CY is singular
is the discriminant locus.  All this is very complicated, but a reasonable proxy for the smallest length scale
on the CY is the shortest distance between the solutions of $\nabla f=0$ in $\IC^5$ and the solutions of $f=0$,
where we normalize $z$ so that the largest distance between a pair of points $f(z_1)=f(z_2)=0$ is unity,
$\max |z_1-z_2|=1$.  This could presumably be estimated numerically.

So, the conjecture is that the error $\max |\eta-1|$ of the CY metric, as a function of moduli, grows as an
inverse power of $L$, and furthermore that this maximum error is attained at the point on the CY closest to
$\nabla f=0$.  Furthermore, the error should decrease with the maximum $k$, which is just $k$ for the 
metrics \eq{a1} but is the degree of the final layer for the networks.

How can we measure the ``complexity'' of the CY?  This is really much less clear and we probably don't want
to do it in this paper.  One idea is that we can average the coefficients of $f$ to get another $\hat f$ with more
discrete symmetry, and measure the distance from $f$ to $\hat f$.  The really correct measure of distance uses
the metric on $\CM_c$, for which the K\"ahler potential is $K = \log\int \Omega\wedge\bar\Omega$.
This could also be estimated numerically, but this is a whole new project.


\iffalse
\section{ Neural Tangent Kernel }

Given a loss function $\CL$, we define the NTK on $M\times M$ as
\be
K_{NTK}(z,z') = \sum_w \Pi(w,w') \vev{ \frac{\partial\CL}{\partial w}\bigg|_z \frac{\partial\CL}{\partial w'}\bigg|_{z'} }.
\ee
where the expectation value is over the initialization distribution for the weights $w$.
Our loss function is the integral of a density on $M$, so the NTK will be a bilocal density,
\be
K_{NTK}(z,z') = d\mu_\Omega(z) d\mu_\Omega(z') \hK_{NTK}(z,z') .
\ee


First consider the loss function \eq{KLobj}, then
\bea
\frac{\partial\CL}{\partial w}\bigg|_z &=& d\mu_\Omega \omega^{-1} \frac{\partial\omega}{\partial w}\bigg|_z \\
&=& d\mu_\Omega \omega^{i\bj} \partial_i \bar\partial_\bj \frac{\partial K}{\partial w}\bigg|_z .
\eea
If we take the linear ansatz \eq{a1}, then $w\cong h$ and this becomes
\bea \label{eq:Lgrad}
\frac{\partial\CL}{\partial w}\bigg|_z &=& d\mu_\Omega \omega^{i\bj} \partial_i \bar\partial_\bj 
\frac{\partial }{\partial w} \log h_{I\bJ} s^I \bs^\bJ  \bigg|_z \\
&=& d\mu_\Omega \omega^{i\bj} \partial_i \bar\partial_\bj 
\frac{s^I \bs^\bJ }{ h_{I\bJ} s^I \bs^\bJ }  \bigg|_z .
\eea
This is already nonlinear in $h$, and the derivatives will make it more nonlinear.
So, even if $h$ is Gaussian initialized, it will not be easy to compute the expectation.

In fact the NTK arguments assume the quadratic loss function \eq{Lquad}, whose
gradient lives in the same space as the target.  In fact the formulas are simpler if we
take the measure $\omega^3$.  Let us take holomorphic weights $a$
acting on $s$, then the gradient is (factor 2?)
\be
\frac{\partial\CL}{\partial a} = \int \left(\frac{\omega^3}{ \Omega \wedge\bar\Omega} - 1\right)
 \omega^2 \wedge \frac{\partial\CL}{\partial a} .
\ee
The gradient of the constant term should give zero, so everything is computable in terms of
a two-point function on the space of weights
\be
\Pi_\mu( a, b ) \equiv \int_M \frac{ \omega^3_a \omega^3_b } { \mu }
\ee
with $\mu=\mu_\Omega$.
This factors through a two-point function on the space of volume forms.

Let the initialization covariance be
\be
\vev{h_{I\bJ} h_{K\bL}} = P_{I\bL} P_{K\bJ} ,
\ee
then this determines a kernel
\be
K_P(z,\bz') = P_{I\bK} s^I(z) \bs^\bK(\bz') .
\ee
If the sum $\Pi$ over $w$ factorizes into a product of holomorphic and antiholomorphic,
then it defines another kernel $K_\Pi$, which could be the same as $K_P$.
This gets us down to
\be
\hK_{NTK}(z,z') = \omega^{i\bj} \partial_i \bar\partial_\bj  \bigg|_z
\omega^{k\bL} \partial_k \bar\partial_\bL \bigg|_{z'} |K_\Pi(z,\bz')|^2 \vev{ e^{-K{(z,\bz)}} e^{-K{(z',\bz')}} }.
\ee
To compute this expectation value, we can write the inverses as (fill in details!)
\bea
 \vev{ e^{-K{(z,\bz)}} e^{-K{(z',\bz')}} } &=& \int_0^\infty dt \int_0^\infty dt' \ \vev{ e^{-ths\bs|_z-t'hs\bs|_z'} } \\
 &=& \int_0^\infty dt \int_0^\infty dt' \ \exp -\half P P\left( t s(z)\bs(\bz) + t' s(z') \bs(\bz') \right)^2 \\
 \nonumber
 &=& \int_0^\infty dt \int_0^\infty dt' \ \exp -\half\left( t^2 |K_P(z,\bz)|^2 + 2tt' |K_P(z,\bz')|^2 + (t')^2|K_P(z',\bz')|^2
 \right) \\
 &\sim& \det\left(\begin{matrix} 
 |K_P(z,\bz)|^2 & |K_P(z,\bz')|^2 \\
 |K_P(z,\bz')|^2 & |K_P(z',\bz')|^2 \end{matrix} \right)^{-1/2}
\eea
This would be doable, but the problem is that $\omega^{-1}$ in \eq{Lgrad} also depends on $h$ in an
even more nonlinear way.  It is probably better to choose a loss function with simpler dependence on $h$.
Still, all formulas for the curvature require inverting the metric.
We may be better off with a ``first order'' formalism which includes the inversion as an explicit condition.

Integration over $h$ is a ``random K\"ahler geometry'' problem in the sense of Klevtsov, Zelditch {\it et al}.
Suppose this works, presumably the NTK will be a natural geometric kernel.  We probably want to set $P=\Pi$
and minimize the dependence on prior data.

Can we guess the result?  We make predictions from the NTK as
\be
y_{pred}(x) = \vec K_{NTK}(x,x_i) \cdot \left(K_{NTK}(x_i,x_j) \right)^{-1} \cdot \vec y_j .
\ee
So, if we had an orthonormal basis in which to expand the holomorphic volume form, we could fit it.
However the goal here is not so much to fit the holomorphic volume form, rather it is to fit it in terms of
a K\"ahler volume form.  Thus it would be more useful to have a $K_\omega$ with simple dependence
on $\omega$ and $h$, such that the minimum of
\be
\CL = \sum_{i} \left(d\mu_\Omega(x_i) - \sum_j K_\omega(x_i,x_j) d\mu_\Omega(x_j)\right)^2
\ee
is at the Ricci flat $\omega$.  A natural candidate or at least building block for
this $K_\omega$ is the $Q$ operator of section 4
of Donaldson's 2005 paper, which is projection on the orthonormal basis of functions $s^I \bs^\bJ/hs\bs$.
If we take $h$ to be the $d\mu_{\Omega,\CD}$ balanced metric, then this should be a reproducing 
kernel (check! also how do we orthonormalize the basis?).  
Conversely, optimizing $\CL$ should lead us to the balanced metric.
We can then use \eq{a1} to obtain $K$.

If this works, then since our multilayer ansatz is a nonlinear subspace of the general space of sections,
then we just need to restrict the optimization to this space.  Now we have an explicit formula for
$hs\bs$, but what does it mean to ``project on a nonlinear subspace'' ?  Does it mean to accept the
final layer defining the vector $s$ as defining the subspace?

The upshot would be that the explicit computation of a loss function like \eq{KLobj} or even \eq{cyobj} may be an
unneeded complication -- since we have the explicit basis of sections we can construct our own kernel
which will also have optimality properties for the Ricci flat metric.  Or at least for the balanced metric, which
if we are primarily interested in the Ricci flat metric, will bring in another $1/k^2$ error.


\section{ Analogies to information geometry, and other generalizations }

An amusing question is whether there is any simple interpretation of the double derivative
$\partial^2/\partial w\bar \partial \bar w$ of either objective function \eq{cyobj} or \eq{KLobj}.
This is somewhat like a K\"ahler metric
on the space of metrics and possibly has this interpretation.

Follow this up and discuss other relations between K\"ahler geometry, Hessian geometry and information geometry.

Possible analogs in quantum information?  We can think of $\Omega$ as a wave function whose norm squared
gives the probability.  Is there an analogous definition for $\omega_g^3$, perhaps involving a density matrix? 
The expression \eq{defFK} gives us $e^K$ as a density matrix for the LLL states, but it's not clear how to think
of the Monge-Ampere expression \eq{MA}.

\subsection{ General Riemannian geometries }

We can generalize the concept to embed a manifold $M$ into a high dimensional space $\IR^N$ using a neural network.
Then, rather than define a single function $K$, we can pull back the Euclidean metric to get a metric on $M$,
and optimize a geometric condition on it.  What class of embeddings can we get this way?

references: Roweis and Saul 2000, Hauser and Ray 2017 ?

Start with a two layer network and ReLU activation.  We first need a primary embedding of $M$ into some $\IR^n$.
Each point $x$ will have a neighborhood which is
linearly embedded, and these will make up regions bounded by planes $M_{ij} x^j=0$.  This representation
will give a PL metric and we need to either smooth it, or define a Ricci flat PL metric.  Either way we will need
to find all the planes and their intersections.  Also the condition that this is an embedding looks nontrivial.
\fi

\appendix

\section{ Math review }

A first reference is GSW volume II chapters 15 and 16, especially \S 15.3, 15.4 and 15.5 on 
the Hodge decomposition and Dolbeault cohomology.
After that, a classic math reference is Griffiths and Harris, Principles of Algebraic Geometry.
This needs serious study, but \S 0.2 on complex manifolds and Dolbeault cohomology
and \S 0.7 on K\"ahler metrics are not hard.

To go deeper into the subject one should study toric geometry and toric hypersurfaces.
The classic physics paper is Witten's hep-th/9301042, and a standard math introduction is \cite{Fulton}.
There are a lot of string theory PhD theses which work this out, e.g. Bouchard  hep-th/0609123.

Let us first give the computation of the volume of $\IC\IP^n$ and a hypersurface $M$ defined as $f=0$ in this space.
These volumes are topological quantities, meaning that they are invariant under continuous deformations
of the metric.  In fact if we take the Fubini-Study metric 
\be \label{eq:normK}
K=\frac{i}{2\pi}\log \sum_i^{n+1} |Z^i|^2
\ee
or its restriction to $M$, normalized as we just did, the volumes will be integers.

The formula \eq{omegag} expresses the K\"ahler metric as a total derivative of the K\"ahler potential.
On reviewing Dolbeault cohomology, one sees a closely related formula
\be \label{eq:defomega}
\omega = \partial\bpartial K
\ee
for the $(1,1)$-form $\omega$, the K\"ahler form.  This can be used to construct the volume form,
\be
\sqrt{\det g} = \det\omega = \frac{1}{n!}\omega^n .
\ee
(see p.31 of GH; it would be nice to check the normalizations in a symbolic algebra system).
Furthermore, since $\partial\omega=0$, we have
\be \label{eq:partialvol}
\omega^n = \partial (\bpartial K\wedge \omega^{n-1}).
\ee

Now if $K$ were a single valued function, \eq{defomega} would imply that
the integral of $\omega$ over any two-cycle would vanish.  It is not, but this argument tells us that varying $K$ by
a function, $K\rightarrow K+f$, does not change the integral of $\omega$.  The same holds for the volume form
by \eq{partialvol}.

Next, let us see that for $K$ given by \eq{normK},
\be\label{eq:dirac}
\int_{\Sigma} \omega \in \IZ,
\ee
for any two-cycle $\Sigma$.  Furthermore there is a two-cycle for which this integral is $1$.
The integral only depends on the homology class of the cycle. 
In fact $\IC\IP^n$ has a unique generator of $H^2$, as
can be seen by embedding $\IC\IP^1$ into it as $(z^1,z^2)\rightarrow(z^1,z^2,0,\ldots,0)$, the fact
that $\IC\IP^1$ is topologically a two-sphere, and the $SU(n+1)$ symmetry of $\IC\IP^n$ which takes
all such embeddings into each other.  Thus we can just compute this integral on $\IC\IP^1$.
You should do this exercise.

A simpler way to do this integral and understand why it is quantized, is to break up $\IC\IP^n$
into patches, and express the integral as a sum of terms involving the differences of $K$ between
different patches.  In mathematics this comes from a relation between $\omega$ and an associated
gauge field and line bundle $\CL$.  In other words, we can postulate a $U(1)$ gauge field $A$, with
curvature $F=dA$, such that
\be
\omega = F .
\ee 
Then, the integral of $F$ over a basis of homology two-cycles, is by definition the first Chern class
of the line bundle.  One writes $c_1(\CL)=1$, and in the math literature this $\CL$ is called $\CO(1)$.
You can read about this in the references, but the simplest physical construction
of $F$ is the following: we consider a Dirac monopole at the origin in $\IR^3$, and identify $\IC\IP^1\cong S^2$
with the sphere of unit radius centered at the origin.  Then, $A$ is the monopole field, and the fact
that the integral \eq{dirac} is quantized is just the Dirac quantization condition (where we set the charge
of the electron to 1).

This is related to our problem as follows.  We can relate the coordinates $Z^i$ to solutions of the
Schr\"odinger equation in the monopole magnetic field on $\IC\IP^1$,
\be
\psi_a^{(i)} = Z_a^i \exp -\half |Z_a|^2 .
\ee
Here $a=1,2$ labels the patches with $Z^a=1$, and $\psi_a$ is the wave function in patch $a$.
$Z^i_a$ is the value of the coordinate $Z^i$ in that patch.
In fact these solutions are ground states in the lowest Landau level.

Similarly, if one put $k$ monopoles at the origin, one would get a solution with magnetic field $k$
times as large.  This defines the line bundle $\CO(k)$, with $c_1=k$.

Now, the math fact which I won't explain in detail, is that the numbers $c_1$ are also related to the
number of zeroes of a section of the bundle.  In the case at hand they are equal, for example $Z^i$ has a single
zero in the patch $a\ne i$ and no zero in the patch $a=i$.  Similarly the section $(Z^i)^k$ of $\CO(k)$ has
$k$ zeroes.  In fact one can turn all the problems of integrating wedge products of these $\omega$'s,
into counting zeroes of simultaneous equations.  This is called intersection theory in the mathematics and
in the simple case at hand works as follows: we can associate $\CO(1)$ with a dual two-cycle $\Sigma$,
the generator of $H^2$.  Then the powers $\omega^k$ are associated with dual $2k$-cycles which each
generate their own $H^{2k}$.  This implies that
\be
\int_{\IC\IP^n} \omega^n = 1 
\ee
and thus the volume of $\IC\IP^n$ in the K\"ahler metric derived from \eq{normK} is $1/n!$.

One can check this directly by writing the coordinates as
\be
Z^i = r_i e^{i\theta_i}
\ee
upon which the condition $1=\sum |Z^i|^2$ becomes $1=\sum r_i^2$.  Thus $\IC\IP^n$ is closely
related to the sphere $S^{2n+1}$ -- it is obtained by quotienting by the overall phase, 
$\theta_i\rightarrow\theta_i+\epsilon$.  This is a circle with constant volume, so the two volumes
are just related by an overall $2\pi$ (I think).

Next, let us consider $M$, the hypersurface $f=0$ in $\IC\IP^{n+1}$, where $f$ is a degree $k$ homogeneous
polynomial.  The volume form on $M$ is $\omega^n/n!$ where $\omega$ is the restriction of
$\omega$ as above to this surface.  To do this restriction in practice, one must embed the $n$-dimensional
cotangent space $T^*M$ in the $n+1$-dimensional cotangent space  $T^*\IC\IP^{n+1}$.  Now the homogeneous
coordinates $Z^i$ are also sections of a line bundle $\CL$, defined by restricting $\CO(1)$ to $M$.
However we cannot use them all as coordinates in a patch as they are redundant.  In terms of the
cotangent bundle we can express this redundancy as
\be
T^*\IC\IP^n \cong T^*M \oplus \CN^*_{\IC\IP^{n+1}} M ,
\ee
where $\CN^*_{\IC\IP^{n+1}} M$ is the conormal bundle, in other words the dual to the normal bundle in 
which vectors normal to $M$ live.  While the normal vector to $M$ depends on the metric, one can choose a section of
the conormal bundle which does not.  It is a one-form which vanishes if we contract it with a tangent vector on $M$.
Thus it is just the one-form $df$, since the condition that a vector $v$ is tangent to $M$ is $v^i\partial_i f=0$.

Thus, to restrict $\omega$ and the metric to $M$, we want to choose a local set of coordinates in which one of
the coordinates $Z^i$ is replaced by $f$, and change basis from $(dZ^1,dZ^2,\ldots,dZ^{n+1})$ to 
$(dZ^1,dZ^2,\ldots,df,\ldots,dZ^{n+1})$, and omit $df$.  To restrict the volume form we just need the Jacobian
of the resulting matrix (with $df$ omitted).  Writing the new coordinates as $\hat Z^i$,
one can see that this matrix is (for $n=2$ and replacing $\hat Z^3$),
\be
\left(\begin{matrix} d\hat Z^1 & d\hat Z^2 & df \end{matrix} \right) = 
\left(\begin{matrix} 1 & 0 & 0 \\
0 & 1 & 0 \\
\frac{\partial f}{\partial Z^1} & \frac{\partial f}{\partial Z^2} & \frac{\partial f}{\partial Z^3} \\
\end{matrix} \right)
\left(\begin{matrix} dZ^1 & dZ^2 & dZ^3 \end{matrix} \right).
\ee
To reexpress $\omega$ in the new frame, one must invert this matrix to get $dZ^i$ in terms of the 
new frame.  One can then omit the $\partial f$ and $\bpartial \bar f$ components.

As we discussed, we want to choose subpatches of each patch on $\IC\IP^{n+1}$ on which
the various components $|\partial f/\partial Z^i|$ are maximized, and then omit this $\hat Z^i$ in that patch.
Thus this relation will take a slightly different form in each patch.

One can even carry out the argument in terms of intersection theory to get the volume of $M$
in the restriction of the Fubini-Study metric.
\iffalse
This follows from the relation between the cotangent bundles on $\IC\IP^{n+1}$ and on $M$,
\be
0 \rightarrow T^*M \rightarrow  T^*{\IC\IP^{n+1} \rightarrow  \CN_{\IC\IP^{n+1}} M \rightarrow 0 .
\ee
This ``exact sequence''
 includes the fact we just used that locally $T^*{\IC\IP^{n+1}$ breaks up into the cotangent
and conormal bundles on $M$, but in fact makes sense on the whole manifold $M$.
It implies a relation between the Chern classes,
\be
c_1(T^*{\IC\IP^{n+1}) = c_1(T^*M) + c_1(\CN_{\IC\IP^{n+1}} M)
\ee
This can be compared with $f$ which
is also a section of the normal bundle.  
(this is explained in the ``adjunction formulas'' in \S 1.1 of GH, pages 145--147 in my copy), which tells us
that $c_1(\CN_{\IC\IP^{n+1}} M) = -\degree f$.  Then one can compute $c_1(T^*{\IC\IP^{n+1})$
to find that
\be
c_1(T^*M) = \mbox{degree} f - (n+2).
\ee
Thus we see the degree required to get $c_1(M)=0$ and a Calabi-Yau metric.
\fi
Now, the K\"ahler form $\omega$ from \eq{normK} is the curvature of $\CO(1)$, which is just
obtained by restriction.  To do an integral over $M$ in intersection theory, one can wedge the
form with the curvature of the normal bundle.  This is $\degree f$ as is 
explained in the ``adjunction formulas'' in \S 1.1 of GH, pages 145--147 in my copy.
Thus we have the volume
\bea
\frac{1}{n!}\int_M \omega^n &=& \frac{\degree f }{n!} \\
&=& \frac{n+2}{n!} \qquad \mbox{if} \ \degree f =n+2
\eea
as is the case for a Ricci flat manifold.  So the volumes for $T^2$, K3 and the quintic should be
3, 2 and $5/6$ respectively.

\section{ Symmetries }

While the general quintic hypersurface has no symmetries, we will restrict attention to the
hypersurfaces
\be \label{eq:fermat}
0 = f(z) = \sum_{i=1}^5 z_i^5 + \psi z_1z_2z_3z_4z_5
\ee
which have $S_5 \ltimes \IZ_5^3$ symmetry (the special case $\psi=0$ has an additional $\IZ_5$).
Let us call this symmetry group $\Gamma$ and denote a general  element as $(\alpha,\sigma)$ where
\be
z_j \rightarrow \exp \frac{2\pi i\alpha_j}{5} \sigma_{j,k} z_k 
\ee
where $\sigma_{j,k}$ is a permutation, then the parameters must satisfy $\sum \alpha_j=0 (5)$ to
preserve $f$.  In addition an overall rotation $z_j\rightarrow e^{2\pi i/5}z_j$ can be compensated
by $\lambda$.   These conditions are satisfied by the following ``charge vectors'' 
\bea
\alpha^{(1)} &=& ( 1, -1, 0, 0, 0 ) \\
\alpha^{(2)} &=& ( 0, 1, -1, 0, 0 ) \\
\alpha^{(3)} &=& ( 0, 0, 1, -1, 0 ) \\
\eea
which can be grouped into a vector $\vec \alpha=(\alpha^{(1)},\alpha^{(2)},\alpha^{(3)})$.

These symmetries must preserve the K\"ahler potential up to an overall constant shift.
This will force many of its parameters to be zero and relate other parameters.

For the general expression  \eq{a1}, this requires every nonzero matrix element $h_{I,\bJ}$ to
transform by the same phase.  Thus a nonzero element must satisfy
\be
0 = \sum_{a=1}^k {\vec \alpha}( I_a ) - {\vec \alpha}( J_a ) 
\ee
Furthermore the $S_k$ action relates many of the matrix elements.

Let us consider some low order cases.  For $k=1$ the $\IZ_5^3$ symmetry sets all the off-diagonal
terms to zero, while the $S_5$ symmetry forces $h$ to be proportional to the identity.
This amounts to the statement that 
\be
X^{1,1} = \sum_{i=1}^5 |z_i|^2
\ee
is the only polynomial of degree $(1,1)$ invariant under $\Gamma$.

For $k=2$ the identity $h$ is of course allowed, but there is an another allowed diagonal $h$.
This is because there are two invariants of degree $(2,2)$, $(X^{1,1})^2$ and
\be
X^{2,2} = \sum_{i=1}^5 |z_i|^4 .
\ee

Clearly we can make a new invariant $X^{l,l}$ at each order $l$, and get invariants 
\be
\prod (X^{1,1})^{n_1}  (X^{2,2})^{n_2} \ldots
\ee
with $\sum_l l n_l = k$.  This is however not a complete set.  This is clear for $k\ge \deg f$
as $f$ contains new invariants, for example $X^{5,0}\equiv z_1z_2z_3z_4z_5$ and its
complex conjugate $X^{0,5}$.

In practice we can use an overcomplete set of parameters, so what we want is a general algorithm
that produces a not too redundant set along with the linear transformation $T$ mapping them into the matrix $h$.
This could be done by first imposing the $\IZ_5^3$ symmetry to set most of the matrix elements to zero,
and then applying the following algorithm.
\begin{enumerate}
\item Create an empty set $L$ which will contain an independent set of matrix elements $h_{I,\bar J}$.
\item Iterate through the matrix elements $h_{I,\bar J}$ which respect $\IZ_5^3$ symmetry.
\item Considering all $\sigma\in S_5$, if no $(\sigma I,\sigma \bar J)$ appears in the set $L$, add $(I,\bar J)$ to $L$.
\item If we add $(I,\bar J)$, the corresponding matrix elements of $T$ are
\be
T_{(\sigma I,\sigma \bar J),(I,\bar J)} = 1 .
\ee
\end{enumerate}
In general this might not find all the relations, but it should be good enough.

For the neural networks, we presumably can only combine quantities with the same symmetry properties
in each layer.  This is quite a drastic restriction, for example in the first layer with inputs $z_i$, the
weight matrix would have to be the identity matrix.  As we go deeper there will be allowed combinations,
but maybe not enough - we might want to generalize $z\rightarrow z^2$ to products of sections with
different degrees.

Another approach this suggests is to use the invariants $X^{l,l}$, $X^{5,0}$ and $X^{0,5}$
as our elementary variables.  A straightforward way to use these is to decompose $(k,k)$ in all possible ways, {\it e.g.}
\bea
Z^{2,2} &=& w_{1,1} (X^{1,1})^2 + w_2 X^{2,2} \\
Z^{3,3} &=& w_{1,1,1} (X^{1,1})^3 + w_{1,2} X^{1,1} X^{2,2} + w_3 X^{3,3} \\
Z^{4,4} &=& w_{1,1,1,1} (X^{1,1})^4 + w_{1,3} X^{1,1} X^{3,3} + w_{2,2} (X^{2,2})^2 +
 w_{1,1,2} (X^{1,1})^2 X^{2,2}+ w_4 X^{4,4} 
\eea
and so on.  But instead of taking $K=Z^{k,k}$ defined this way, we could also use some of the $Z$'s
as intermediate layers.  For example, we could exclude all the terms of more than quadratic order
in the $X$'s, instead using the $Z$'s:
\bea
Z^{3,3} &=& w_{1,2} X^{1,1} Z^{2,2} + w_3 X^{3,3} \\
Z^{4,4} &=& w_{1,3} X^{1,1} Z^{3,3} + w_{2,2} (Z^{2,2})^2 + w_4 X^{4,4} \\
Z^{5,5} &=& w_{1,4} X^{1,1} Z^{4,4} + w_{2,3} Z^{2,2} Z^{3,3} + w_5 X^{5,5} + \tilde w X^{5,0} X^{0,5}
\eea
so that the result would depend on nonlinear combinations of the weights $w$.
With this approach we could get a very high degree function with $O(k^2)$ weights.

\def\bx{\bf}
\def\etal{{\it et.~al.}}
\def\NP#1{Nucl. Phys. {\bx #1}}
\def\NPPS#1{Nucl. Phys. (Proc. Suppl.) {\bx #1}}
\def\PL#1{Phys. Lett. {\bx #1}}
\def\PR#1{Phys. Rev. {\bx #1}}
\def\PRL#1{Phys. Rev. Lett. {\bx #1}}
\def\ANN#1{Annals Phys. {\bx #1}}
\def\IJMP#1{Int. J. Mod. Phys. {\bx #1}}
\def\JH#1{J. High Energy Phys. {\bx #1}}
\def\JMP#1{J. Math. Phys. {\bx #1}}
\def\RMP#1{Rev. Mod. Phys. {\bx #1}}
\def\CMP#1{Comm. Math. Phys. {\bx #1}}
\def\ATMP#1{Adv. Theor. Math. Phys {\bx #1}}
\def\ep#1{{#1}}

\bibliographystyle{amsplain}
\begin{thebibliography}{10}

\bibitem{CYreview}
``Calabi-Yau metrics for string compactification,''
\NP{B898} (2015) 667-674,
\ep{ arXiv:1503.02899}.

\bibitem{MS1}
Hori, K., Thomas, R., Katz, S., Vafa, C., Pandharipande, R., Klemm, A., Vakil, R. and Zaslow, E., 2003. 
{\it Mirror symmetry} (Vol. 1). AMS

\bibitem{MS2}
Aspinwall, P., Bridgeland, T., Craw, A., Douglas, M. R., 
Gross, M., Kapustin, A., Moore, G. W., Segal, G.,
Szendroi, B. and Wilson, P. M. H., 2009. {\it Dirichlet branes and mirror symmetry} (Vol. 4). AMS
  
    %\cite{Gaiotto:2008cd}
\bibitem{Gaiotto:2008cd} 
  D.~Gaiotto, G.~W.~Moore and A.~Neitzke,
  ``Four-dimensional wall-crossing via three-dimensional field theory,''
  Commun.\ Math.\ Phys.\  {\bf 299}, 163 (2010)
 [arXiv:0807.4723 [hep-th]].

%\cite{Kachru:2020tat}
\bibitem{Kachru:2020tat}
S.~Kachru, A.~Tripathy and M.~Zimet,
``K3 metrics,''
[arXiv:2006.02435 [hep-th]].
%0 citations counted in INSPIRE as of 07 Jun 2020

  %\cite{Headrick:2005ch}
\bibitem{Headrick:2005ch} 
  M.~Headrick and T.~Wiseman,
  ``Numerical Ricci-flat metrics on K3,''
  Class.\ Quant.\ Grav.\  {\bf 22}, 4931 (2005)
  [hep-th/0506129].

\bibitem{Donaldson}
S. K. Donaldson, 
``Some numerical results in complex differential geometry,'' [math.DG/0512625].


  %\cite{Douglas:2006hz}
\bibitem{Douglas:2006hz} 
  M.~R.~Douglas, R.~L.~Karp, S.~Lukic and R.~Reinbacher,
  ``Numerical solution to the hermitian Yang-Mills equation on the Fermat quintic,''
  JHEP {\bf 0712}, 083 (2007)
  [hep-th/0606261].

%\cite{Douglas:2006rr}
\bibitem{Douglas:2006rr} 
  M.~R.~Douglas, R.~L.~Karp, S.~Lukic and R.~Reinbacher,
  ``Numerical Calabi-Yau metrics,''
  J.\ Math.\ Phys.\  {\bf 49}, 032302 (2008)
  [hep-th/0612075].

%\cite{Doran:2007zn}
\bibitem{Doran:2007zn} 
  C.~Doran, M.~Headrick, C.~P.~Herzog, J.~Kantor and T.~Wiseman,
  ``Numerical Kahler-Einstein metric on the third del Pezzo,''
  Commun.\ Math.\ Phys.\  {\bf 282}, 357 (2008)
  [hep-th/0703057 [HEP-TH]].

  %\cite{Braun:2007sn}
\bibitem{Braun:2007sn} 
  V.~Braun, T.~Brelidze, M.~R.~Douglas and B.~A.~Ovrut,
  ``Calabi-Yau Metrics for Quotients and Complete Intersections,''
  JHEP {\bf 0805}, 080 (2008)
  [arXiv:0712.3563 [hep-th]].

\bibitem{Bunch:2008}
 R.~S.~Bunch and S.~K.~Donaldson,
  ``Numerical approximations to extremal metrics on toric surfaces,''
  {\it Handbook of geometric analysis} No. 1, 1-28,
 Adv. Lect. Math. (ALM), 7, Int. Press, Somerville, MA, 2008, 
 [arXiv:0803.0987 [math.DG]].

\bibitem{Seyyedali:2008}
 R.~Seyyedali,
   ``Numerical Algorithms for Finding Balanced Metrics on Vector Bundles,''
   Asian J. Math. 13 (2009), no. 3, 311-321, 
   [arXiv:0804.4005 [math.DG]].

  %\cite{Braun:2008jp}
\bibitem{Braun:2008jp} 
  V.~Braun, T.~Brelidze, M.~R.~Douglas and B.~A.~Ovrut,
  ``Eigenvalues and Eigenfunctions of the Scalar Laplace Operator on Calabi-Yau Manifolds,''
  JHEP {\bf 0807}, 120 (2008)
  [arXiv:0805.3689 [hep-th]].
    
%\cite{Anderson:2009ge}
\bibitem{Anderson:2009ge} 
  L.~B.~Anderson, J.~Gray, D.~Grayson, Y.~H.~He and A.~Lukas,
  ``Yukawa Couplings in Heterotic Compactification,''
  Commun.\ Math.\ Phys.\  {\bf 297}, 95 (2010)
  [arXiv:0904.2186 [hep-th]].

  %\cite{Headrick:2009jz}
\bibitem{Headrick:2009jz} 
  M.~Headrick and A.~Nassar,
  ``Energy functionals for Calabi-Yau metrics,''
  Adv.\ Theor.\ Math.\ Phys.\  {\bf 17}, 867 (2013)
  [arXiv:0908.2635 [hep-th]].
  %%CITATION = ARXIV:0908.2635;%%

%\cite{Anderson:2010ke}
\bibitem{Anderson:2010ke} 
  L.~B.~Anderson, V.~Braun, R.~L.~Karp and B.~A.~Ovrut,
  ``Numerical Hermitian Yang-Mills Connections and Vector Bundle Stability in Heterotic Theories,''
  JHEP {\bf 1006}, 107 (2010)
  [arXiv:1004.4399 [hep-th]].

  %\cite{Anderson:2011ed}
\bibitem{Anderson:2011ed} 
  L.~B.~Anderson, V.~Braun and B.~A.~Ovrut,
  ``Numerical Hermitian Yang-Mills Connections and Kahler Cone Substructure,''
  JHEP {\bf 1201}, 014 (2012)
  [arXiv:1103.3041 [hep-th]].
  
%\cite{Cui:2019uhy}
\bibitem{Cui:2019uhy}
W.~Cui and J.~Gray,
%``Numerical Metrics, Curvature Expansions and Calabi-Yau Manifolds,''
JHEP \textbf{05}, 044 (2020)
doi:10.1007/JHEP05(2020)044
[arXiv:1912.11068 [hep-th]].
%0 citations counted in INSPIRE as of 01 Jun 2020

\bibitem{Hoyer2019}
  Hoyer, S., Sohl-Dickstein, J. and Greydanus, S., 2019. Neural reparameterization improves structural optimization. 
   arXiv:1909.04240.
  
  %\cite{Ashmore:2019wzb}
\bibitem{Ashmore:2019wzb}
A.~Ashmore, Y.~H.~He and B.~A.~Ovrut,
%``Machine learning Calabi-Yau metrics,''
[arXiv:1910.08605 [hep-th]].
%5 citations counted in INSPIRE as of 01 Jun 2020

\bibitem{Anderson:2020}
L. B. Anderson, J. Gray, S. Krippendorf, N. Raghuram and F. Ruehle, “Machine learning SU(3) structure,” To appear.

\bibitem{Chrysos2020}
Chrysos, G.G., Moschoglou, S., Bouritsas, G., Deng, J., Panagakis, Y. and Zafeiriou, S., 2020. $\Pi-$ nets: Deep Polynomial Neural Networks.  arXiv:2003.03828.

\bibitem{Mannelli2020}
Mannelli, S.S., Vanden-Eijnden, E. and Zdeborov\'a, L., 2020. Optimization and Generalization of Shallow Neural Networks with Quadratic Activation Functions.  arXiv:2006.15459.

\bibitem{Douglas2020}
Douglas, M.R., Holomorphic feedforward networks, submitted to the
Pure and Applied Mathematics Quarterly special issue in honor of Professor Bernie Shiffman.

\bibitem{Ioffe2015}
Ioffe, S. and Szegedy, C., 2015. Batch normalization: Accelerating deep network training by reducing internal covariate shift. arXiv:1502.03167.

\bibitem{Candelas1990}
Candelas, P. and De la Ossa, X.C., Comments on conifolds 1990 Nucl. Phys. B, 342, p.246.

\bibitem{GH}
Griffiths, P. and Harris, J., 2014. {\it Principles of algebraic geometry}. John Wiley \& Sons.

\bibitem{Huybrechts}
Huybrechts, D., 2006. {\it Complex geometry: an introduction}. Springer Science \& Business Media.

\bibitem{Bishop}
Bishop, C.M., 2006. {\it Pattern recognition and machine learning}. Springer.

\bibitem{Douglas:2008pz} 
  M.~R.~Douglas and S.~Klevtsov,
  ``Bergman Kernel from Path Integral,''
  Commun.\ Math.\ Phys.\  {\bf 293}, 205 (2010)
  [arXiv:0808.2451 [hep-th]].

\bibitem{Cybenko}
Cybenko, G., 1989. Approximation by superpositions of a sigmoidal function. Mathematics of control, signals and systems, 2(4), pp.303-314.

\bibitem{TFhessian}
{\tt https://github.com/tensorflow/tensorflow/blob/r2.0/tensorflow/python/ops/gradients\_impl.py}

\bibitem{Fulton}
{\tt http://home.ustc.edu.cn/~hanzr/pdf/Introduction\%20to\%20Toric\%20Varieties-Fulton.pdf}


\end{thebibliography}

\end{document}
%%%%%%%%%%%%%%%%%%%%%%%%%%%%%%%%%%%%%%%%%%%%%%%%%%%%%%%%%%%%%%%%%%





