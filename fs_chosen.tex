\documentclass[11pt]{article}
\usepackage{geometry,amsmath}                % See geometry.pdf to learn the layout options. 
\geometry{letterpaper}                   % ... or a4paper or a5paper or ... 
%\geometry{landscape}                % Activate for for rotated page geometry
%\usepackage[parfill]{parskip}    % Activate to begin paragraphs with an empty line rather than an indent
\usepackage{graphicx}
\usepackage{amssymb}
\usepackage{epstopdf}
%\DeclareGraphicsRule{.tif}{png}{.png}{`convert #1 `dirname #1`/`basename #1 .tif`.png}

\title{Quintics Chosen for CY Project}
%\author{}
%\date{}                                           % Activate to display a given date or no date

\begin{document}
\maketitle
%\section{}
%\subsection{}
The following $f$'s were chosen to represent various quintics for our study.

$$f_0 = z_0^5 + z_1^5 + z_2^5 + z_3^5 + z_4^5 + \psi z_0z_1z_2z_3z_4$$

$$f_1 = z_0^5 + z_1^5 + z_2^5 + z_3^5 + z_4^5 + \psi z_0z_1z_2z_3z_4 +\phi (z_3z_4^4 + z_3^2z_4^3 + z_3^3z_4^2 + z_3^4z_4)$$

$$f_2 = z_3 g(z) + z_4 h(z)$$

$$g = z_0^4 + z_1^4 + z_2^4 + z_3^4 + a z_0z_1z_2z_3$$

$$h = z_0^4 + z_1^4 + z_2^4 + z_4^4 + b z_0z_1z_2z_4$$

\end{document}  
